\documentclass[a4paper, 13pt]{article}
\usepackage[utf8]{inputenc}
\usepackage[T1]{fontenc}
\usepackage{lmodern}
\usepackage{graphicx}
\usepackage{setspace} %pour changer la taille de l'interlignage
\usepackage{amsmath}
\usepackage{array} %débloque des options supplémentaires pour les tableaux
\usepackage[english, french]{babel} %babel est à placer en dernier


\title{Factory Disk List Emu - Emax}
\author{Clement Thomas}
\date{February 2025}

\begin{document}

\maketitle

\vspace{3cm}
\section*{Français}
\vspace{1cm}
\subsection*{ZD 743}
\begin{itemize}
    \item \texttt{00} $\rightarrow$ harpe pas géniale $\rightarrow$ \Large{$\frac{5}{20}$} \normalsize \vspace{0.2cm}
    \item \texttt{01} $\rightarrow$ glissando de harpe \\ peut servir d'effet mais pas génial $\rightarrow$ \Large{$\frac{7}{20}$} \normalsize\vspace{0.2cm}
    \item \texttt{02} $ \rightarrow$ pareil mais chorus $\rightarrow$ \Large{$\frac{7}{20}$} \normalsize \vspace{0.2cm}
    \item \texttt{03} $ \rightarrow$ chorus en stéréo $\rightarrow$ \Large{$\frac{9}{20}$} \normalsize \vspace{0.2cm}
    \item \texttt{04} $ \rightarrow$ synthétiseur d'ambiance avec beaucoup d'attack. Attention, les notes sont des accords $\rightarrow$ \Large{$\frac{5}{20}$} \normalsize \vspace{0.2cm}
    \item \texttt{05} $ \rightarrow$ comme \texttt{03} mais en stéréo et réverbération $\rightarrow$ \Large{$\frac{10}{20}$} \normalsize \vspace{0.2cm}
    \item \texttt{06} $ \rightarrow$ harpe pas géniale $\rightarrow$ \Large{$\frac{7}{20}$} \normalsize \vspace{0.2cm}
    \item \texttt{07} $ \rightarrow$ pareil que \texttt{07} mais avec chorus$\rightarrow$ \Large{$\frac{7}{20}$} \normalsize \vspace{0.2cm}
\end{itemize}


\vspace{2cm}
\subsection*{ZD 744}
\begin{itemize}
    \item \texttt{00} $\rightarrow$ percussion utilisable en effets mais pas exceptionnelle $\rightarrow$ \Large{$\frac{10}{20}$} \normalsize \vspace{0.2cm}
    \item \texttt{01} $\rightarrow$ karimba pas trop mauvais mais peu utilisable en pratique $\rightarrow$ \Large{$\frac{11}{20}$} \normalsize \vspace{0.2cm}
    \item \texttt{02} $\rightarrow$ comme \texttt{00}, avec des sons différents $\rightarrow$ \Large{$\frac{10}{20}$} \normalsize \vspace{0.2cm}
    \item \texttt{03} $\rightarrow$ karimba avec beaucoup d'attack. \\ PAD qui sonne comme un orgue de verre $\rightarrow$ \Large{$\frac{14}{20}$} \normalsize \vspace{0.2cm}
    \item \texttt{04} $\rightarrow$ pareil que \texttt{03} mais avec un léger LFO $\rightarrow$ \Large{$\frac{14}{20}$} \normalsize \vspace{0.2cm}
    \item \texttt{05} $\rightarrow$ mélange de \texttt{01} et de \texttt{01} $\rightarrow$ \Large{$\frac{12}{20}$} \normalsize \vspace{0.2cm}
    \item \texttt{06} $\rightarrow$ mélange de \texttt{00} et \texttt{01} $\rightarrow$ \Large{$\frac{10}{20}$} \normalsize \vspace{0.2cm}
\end{itemize}


\vspace{1.5cm}
\subsection*{ZD 745}
\begin{itemize}
    \item \texttt{00} $\rightarrow$ brass pas géniale $\rightarrow$ \Large{$\frac{8}{20}$} \normalsize \vspace{0.2cm}
    \item \texttt{01} $\rightarrow$ idem $\rightarrow$ \Large{$\frac{8}{20}$} \normalsize \vspace{0.2cm}
    \item \texttt{02} $\rightarrow$ orgue mono avec bon son \\ utile pour les basses longues $\rightarrow$ \Large{$\frac{15}{20}$} \normalsize \vspace{0.2cm}
    \item \texttt{03} $\rightarrow$ idem $\rightarrow$ \Large{$\frac{15}{20}$} \normalsize \vspace{0.2cm}
    \item \texttt{04} $\rightarrow$ comme \texttt{00} mais avec pan sur les touches $\rightarrow$ \Large{$\frac{9}{20}$} \normalsize \vspace{0.2cm}
    \item \texttt{05} $\rightarrow$ orgue stéréo de très bonne qualité \\
    basse moins synthétique que \texttt{02} mais très bonne aussi $\rightarrow$ \Large{$\frac{10}{20}$} \normalsize \vspace{0.2cm}
    \item \texttt{06} $\rightarrow$ \texttt{05} avec pan sur les touches $\rightarrow$ \Large{$\frac{14}{20}$} \normalsize \vspace{0.2cm}
    \item \texttt{07} $\rightarrow$ orgue PAD avec LFO stéréo \\ bon pour créer de l'atmosphère $\rightarrow$ \Large{$\frac{15}{20}$} \normalsize \vspace{0.2cm}
    \item \texttt{08} $\rightarrow$ \texttt{06} avec release $\rightarrow$ \Large{$\frac{14}{20}$} \normalsize \vspace{0.2cm}
    \item \texttt{09} $\rightarrow$ \texttt{09} avec de l'attack sur le filtre $\rightarrow$ \Large{$\frac{15}{20}$} \normalsize \vspace{0.2cm}
    \item \texttt{10} $\rightarrow$ mélange de \texttt{00} et de \texttt{05} en stéréo $\rightarrow$ \Large{$\frac{12}{20}$} \normalsize \vspace{0.2cm}
    \item \texttt{11} $\rightarrow$ \texttt{10} en dual $\rightarrow$ \Large{$\frac{10}{20}$} \normalsize \vspace{0.2cm}
    \item \texttt{00} $\rightarrow$ \texttt{02} modifié ici, filtre dominant \\ son électrique/distortion $\rightarrow$ \Large{$\frac{10}{20}$} \normalsize \vspace{0.2cm}
\end{itemize}


\vspace{0cm}
\subsection*{ZD 746}
\begin{itemize}
    \item \texttt{00} $\rightarrow$ marimba bottle ressemblant au SquareLead \\ 
    peut être intéressant à utiliser $\rightarrow$ \Large{$\frac{13}{20}$} \normalsize \vspace{0.2cm}
    \item \texttt{01} $\rightarrow$ synthétiseur lead avec de bonnes basses sèches. Multi usage $\rightarrow$ \Large{$\frac{14}{20}$} \normalsize \vspace{0.2cm}
    \item \texttt{02} $\rightarrow$ \texttt{01} avec pan sur les touches $\rightarrow$ \Large{$\frac{15}{20}$} \normalsize \vspace{0.2cm}
    \item \texttt{03} $\rightarrow$ \texttt{01} en stéréo $\rightarrow$ \Large{$\frac{16}{20}$} \normalsize \vspace{0.2cm}
    \item \texttt{04} $\rightarrow$ comme \texttt{00} $\rightarrow$ \Large{$\frac{13}{20}$} \normalsize \vspace{0.2cm}
    \item \texttt{05} $\rightarrow$ \texttt{04} avec pan sur les touches $\rightarrow$ \Large{$\frac{12}{20}$} \normalsize \vspace{0.2cm}
    \item \texttt{06} $\rightarrow$ \texttt{04} en stéréo $\rightarrow$ \Large{$\frac{14}{20}$} \normalsize \vspace{0.2cm}
    \item \texttt{07} $\rightarrow$ \texttt{01} avec \texttt{00} stéréo $\rightarrow$ \Large{$\frac{15}{20}$} \normalsize \vspace{0.2cm}
    \item \texttt{08} $\rightarrow$ \texttt{02} + \texttt{05} avec arpège $\rightarrow$ \Large{$\frac{13}{20}$} \normalsize \vspace{0.2cm}
    \item \texttt{09} $\rightarrow$ \texttt{07} en dual $\rightarrow$ \Large{$\frac{15}{20}$} \normalsize \vspace{0.2cm}
\end{itemize}


\vspace{0.5cm}
\subsection*{ZD 747 $\rightarrow$ percussion}
\begin{itemize}
    \item \texttt{00} $\rightarrow$ grosse caisse puissance, caisse claire clappée \\ et intéressant. Sinon nul. Stéréo. $\rightarrow$ \Large{$\frac{12}{20}$} \normalsize \vspace{0.2cm}
    \item \texttt{01} $\rightarrow$ pareil que \texttt{00} mais mono avec pan et plus de notes $\rightarrow$ \Large{$\frac{10}{20}$} \normalsize \vspace{0.2cm}
    \item \texttt{02} $\rightarrow$ cymbales mono avec beaucoup de réverbération \\ utile pour créer une atmosphère $\rightarrow$ \Large{$\frac{11}{20}$} \normalsize \vspace{0.2cm}
    \item \texttt{03} $\rightarrow$ \texttt{02} avec délai et stéréo $\rightarrow$ \Large{$\frac{14}{20}$} \normalsize \vspace{0.2cm}
    \item \texttt{04} $\rightarrow$ grosse caisse pas très utilisable $\rightarrow$ \Large{$\frac{07}{20}$} \normalsize \vspace{0.2cm}
    \item \texttt{05} $\rightarrow$ \texttt{04} mais plus percutant $\rightarrow$ \Large{$\frac{11}{20}$} \normalsize \vspace{0.2cm}
    \item \texttt{06} $\rightarrow$ percussion bois de \texttt{04}, stylé \\ fais des merveilles lorsqu'on superpose 3 octaves $\rightarrow$ \Large{$\frac{14}{20}$} \normalsize \vspace{0.2cm}
    \item \texttt{07} $\rightarrow$ \texttt{06} mais stéréo et légère réverbération $\rightarrow$ \Large{$\frac{17}{20}$} \normalsize \vspace{0.2cm}
    \item \texttt{08} $\rightarrow$ \texttt{07} avec une empreinte stéréo différente $\rightarrow$ \Large{$\frac{17}{20}$} \normalsize \vspace{0.2cm}
    \item \texttt{09} $\rightarrow$ \texttt{01} en stéréo $\rightarrow$ \Large{$\frac{13}{20}$} \normalsize \vspace{0.2cm}
    \item \texttt{10} $\rightarrow$ pareil que \texttt{01} $\rightarrow$ \Large{$\frac{11}{20}$} \normalsize \vspace{0.2cm}
\end{itemize}


\vspace{1cm}
\subsection*{ZD 748}
\begin{itemize}
    \item \texttt{00} $\rightarrow$ bon synthétiseur LEAD mais basse qualité $\rightarrow$ \Large{$\frac{11}{20}$} \normalsize \vspace{0.2cm}
    \item \texttt{01} $\rightarrow$ \texttt{00} avec LFO stéréo + vibrato $\rightarrow$ \Large{$\frac{11}{20}$} \normalsize \vspace{0.2cm}
    \item \texttt{02} $\rightarrow$ \texttt{01} avec pan + beaucoup d'attack \\ et de release \\ $\rightarrow$ \Large{$\frac{15}{20}$} \normalsize \vspace{0.2cm}
    \item \texttt{03} $\rightarrow$ LEAD très sec, avec pan, sympa à utiliser \\ pour mélodie sèche $\rightarrow$ \Large{$\frac{15}{20}$} \normalsize \vspace{0.2cm}
    \item \texttt{04} $\rightarrow$ PAD mono LFO avec pan, attack longue, \\très léger mais release à réduire \\très bonne basse $\rightarrow$ \Large{$\frac{16}{20}$} \normalsize \vspace{0.2cm}
    \item \texttt{05} $\rightarrow$ attack du filtre violente mais sympa. Peut \\surtout être utilisé en mélodie secondaire \\ fait penser à FOREVER YOUNG $\rightarrow$ \Large{$\frac{16}{20}$} \normalsize \vspace{0.2cm}
    \item \texttt{06} $\rightarrow$ \texttt{00} mais en stéréo $\rightarrow$ \Large{$\frac{13}{20}$} \normalsize \vspace{0.2cm}
    \item \texttt{07} $\rightarrow$ \texttt{06} mais en dual mono $\rightarrow$ \Large{$\frac{10}{20}$} \normalsize \vspace{0.2cm}
\end{itemize}


\vspace{1cm}
\section*{English}

\end{document}

%   \texttt{00} $\rightarrow$  $\rightarrow$ \Large{$\frac{10}{20}$} \normalsize \vspace{0.2cm}