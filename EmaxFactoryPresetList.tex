\documentclass[a4paper, 13pt]{article}
\usepackage[utf8]{inputenc}
\usepackage[T1]{fontenc}
\usepackage{graphicx}
\usepackage{setspace} %pour changer la taille de l'interlignage
\usepackage{amsmath}
\usepackage{array} %débloque des options supplémentaires pour les tableaux
\usepackage[english, french]{babel} %babel est à placer en dernier


\title{Factory Disk List Emu - Emax}
\author{Clement Thomas}
\date{February 2025}

\begin{document}

\maketitle

\vspace{3cm}
\section*{Français}
\vspace{1cm}
\subsection*{ZD 743}
\begin{itemize}
    \item \texttt{00} $\rightarrow$ harpe pas géniale $\rightarrow$ \Large{$\frac{5}{20}$} \normalsize \vspace{0.2cm}
    \item \texttt{01} $\rightarrow$ glissando de harpe \\ peut servir d'effet mais pas génial $\rightarrow$ \Large{$\frac{7}{20}$} \normalsize\vspace{0.2cm}
    \item \texttt{02} $ \rightarrow$ pareil mais chorus $\rightarrow$ \Large{$\frac{7}{20}$} \normalsize \vspace{0.2cm}
    \item \texttt{03} $ \rightarrow$ chorus en stéréo $\rightarrow$ \Large{$\frac{9}{20}$} \normalsize \vspace{0.2cm}
    \item \texttt{04} $ \rightarrow$ synthétiseur d'ambiance avec beaucoup d'attack. Attention, les notes sont des accords $\rightarrow$ \Large{$\frac{5}{20}$} \normalsize \vspace{0.2cm}
    \item \texttt{05} $ \rightarrow$ comme \texttt{03} mais en stéréo et réverbération $\rightarrow$ \Large{$\frac{10}{20}$} \normalsize \vspace{0.2cm}
    \item \texttt{06} $ \rightarrow$ harpe pas géniale $\rightarrow$ \Large{$\frac{7}{20}$} \normalsize \vspace{0.2cm}
    \item \texttt{07} $ \rightarrow$ pareil que \texttt{07} mais avec chorus$\rightarrow$ \Large{$\frac{7}{20}$} \normalsize \vspace{0.2cm}
\end{itemize}


\vspace{2cm}
\subsection*{ZD 744}
\begin{itemize}
    \item \texttt{00} $\rightarrow$ percussion utilisable en effets mais pas exceptionnelle $\rightarrow$ \Large{$\frac{10}{20}$} \normalsize \vspace{0.2cm}
    \item \texttt{01} $\rightarrow$ karimba pas trop mauvais mais peu utilisable en pratique $\rightarrow$ \Large{$\frac{11}{20}$} \normalsize \vspace{0.2cm}
    \item \texttt{02} $\rightarrow$ comme \texttt{00}, avec des sons différents $\rightarrow$ \Large{$\frac{10}{20}$} \normalsize \vspace{0.2cm}
    \item \texttt{03} $\rightarrow$ karimba avec beaucoup d'attack. \\ PAD qui sonne comme un orgue de verre $\rightarrow$ \Large{$\frac{14}{20}$} \normalsize \vspace{0.2cm}
    \item \texttt{04} $\rightarrow$ pareil que \texttt{03} mais avec un léger LFO $\rightarrow$ \Large{$\frac{14}{20}$} \normalsize \vspace{0.2cm}
    \item \texttt{05} $\rightarrow$ mélange de \texttt{01} et de \texttt{01} $\rightarrow$ \Large{$\frac{12}{20}$} \normalsize \vspace{0.2cm}
    \item \texttt{06} $\rightarrow$ mélange de \texttt{00} et \texttt{01} $\rightarrow$ \Large{$\frac{10}{20}$} \normalsize \vspace{0.2cm}
\end{itemize}


\vspace{1.5cm}
\subsection*{ZD 745}
\begin{itemize}
    \item \texttt{00} $\rightarrow$ brass pas géniale $\rightarrow$ \Large{$\frac{8}{20}$} \normalsize \vspace{0.2cm}
    \item \texttt{01} $\rightarrow$ idem $\rightarrow$ \Large{$\frac{8}{20}$} \normalsize \vspace{0.2cm}
    \item \texttt{02} $\rightarrow$ orgue mono avec bon son \\ utile pour les basses longues $\rightarrow$ \Large{$\frac{15}{20}$} \normalsize \vspace{0.2cm}
    \item \texttt{03} $\rightarrow$ idem $\rightarrow$ \Large{$\frac{15}{20}$} \normalsize \vspace{0.2cm}
    \item \texttt{04} $\rightarrow$ comme \texttt{00} mais avec pan sur les touches $\rightarrow$ \Large{$\frac{9}{20}$} \normalsize \vspace{0.2cm}
    \item \texttt{05} $\rightarrow$ orgue stéréo de très bonne qualité \\
    basse moins synthétique que \texttt{02} mais très bonne aussi $\rightarrow$ \Large{$\frac{10}{20}$} \normalsize \vspace{0.2cm}
    \item \texttt{06} $\rightarrow$ \texttt{05} avec pan sur les touches $\rightarrow$ \Large{$\frac{14}{20}$} \normalsize \vspace{0.2cm}
    \item \texttt{07} $\rightarrow$ orgue PAD avec LFO stéréo \\ bon pour créer de l'atmosphère $\rightarrow$ \Large{$\frac{15}{20}$} \normalsize \vspace{0.2cm}
    \item \texttt{08} $\rightarrow$ \texttt{06} avec release $\rightarrow$ \Large{$\frac{14}{20}$} \normalsize \vspace{0.2cm}
    \item \texttt{09} $\rightarrow$ \texttt{09} avec de l'attack sur le filtre $\rightarrow$ \Large{$\frac{15}{20}$} \normalsize \vspace{0.2cm}
    \item \texttt{10} $\rightarrow$ mélange de \texttt{00} et de \texttt{05} en stéréo $\rightarrow$ \Large{$\frac{12}{20}$} \normalsize \vspace{0.2cm}
    \item \texttt{11} $\rightarrow$ \texttt{10} en dual $\rightarrow$ \Large{$\frac{10}{20}$} \normalsize \vspace{0.2cm}
    \item \texttt{00} $\rightarrow$ \texttt{02} modifié ici, filtre dominant \\ son électrique/distortion $\rightarrow$ \Large{$\frac{10}{20}$} \normalsize \vspace{0.2cm}
\end{itemize}


\vspace{0cm}
\subsection*{ZD 746}
\begin{itemize}
    \item \texttt{00} $\rightarrow$ marimba bottle ressemblant au SquareLead \\ 
    peut être intéressant à utiliser $\rightarrow$ \Large{$\frac{13}{20}$} \normalsize \vspace{0.2cm}
    \item \texttt{01} $\rightarrow$ synthétiseur lead avec de bonnes basses sèches. Multi usage $\rightarrow$ \Large{$\frac{14}{20}$} \normalsize \vspace{0.2cm}
    \item \texttt{02} $\rightarrow$ \texttt{01} avec pan sur les touches $\rightarrow$ \Large{$\frac{15}{20}$} \normalsize \vspace{0.2cm}
    \item \texttt{03} $\rightarrow$ \texttt{01} en stéréo $\rightarrow$ \Large{$\frac{16}{20}$} \normalsize \vspace{0.2cm}
    \item \texttt{04} $\rightarrow$ comme \texttt{00} $\rightarrow$ \Large{$\frac{13}{20}$} \normalsize \vspace{0.2cm}
    \item \texttt{05} $\rightarrow$ \texttt{04} avec pan sur les touches $\rightarrow$ \Large{$\frac{12}{20}$} \normalsize \vspace{0.2cm}
    \item \texttt{06} $\rightarrow$ \texttt{04} en stéréo $\rightarrow$ \Large{$\frac{14}{20}$} \normalsize \vspace{0.2cm}
    \item \texttt{07} $\rightarrow$ \texttt{01} avec \texttt{00} stéréo $\rightarrow$ \Large{$\frac{15}{20}$} \normalsize \vspace{0.2cm}
    \item \texttt{08} $\rightarrow$ \texttt{02} + \texttt{05} avec arpège $\rightarrow$ \Large{$\frac{13}{20}$} \normalsize \vspace{0.2cm}
    \item \texttt{09} $\rightarrow$ \texttt{07} en dual $\rightarrow$ \Large{$\frac{15}{20}$} \normalsize \vspace{0.2cm}
\end{itemize}


\vspace{0.5cm}
\subsection*{ZD 747 $\rightarrow$ percussion}
\begin{itemize}
    \item \texttt{00} $\rightarrow$ grosse caisse puissance, caisse claire clappée \\ et intéressant. Sinon nul. Stéréo. $\rightarrow$ \Large{$\frac{12}{20}$} \normalsize \vspace{0.2cm}
    \item \texttt{01} $\rightarrow$ pareil que \texttt{00} mais mono avec pan et plus de notes $\rightarrow$ \Large{$\frac{10}{20}$} \normalsize \vspace{0.2cm}
    \item \texttt{02} $\rightarrow$ cymbales mono avec beaucoup de réverbération \\ utile pour créer une atmosphère $\rightarrow$ \Large{$\frac{11}{20}$} \normalsize \vspace{0.2cm}
    \item \texttt{03} $\rightarrow$ \texttt{02} avec délai et stéréo $\rightarrow$ \Large{$\frac{14}{20}$} \normalsize \vspace{0.2cm}
    \item \texttt{04} $\rightarrow$ grosse caisse pas très utilisable $\rightarrow$ \Large{$\frac{07}{20}$} \normalsize \vspace{0.2cm}
    \item \texttt{05} $\rightarrow$ \texttt{04} mais plus percutant $\rightarrow$ \Large{$\frac{11}{20}$} \normalsize \vspace{0.2cm}
    \item \texttt{06} $\rightarrow$ percussion bois de \texttt{04}, stylé \\ fais des merveilles lorsqu'on superpose 3 octaves $\rightarrow$ \Large{$\frac{14}{20}$} \normalsize \vspace{0.2cm}
    \item \texttt{07} $\rightarrow$ \texttt{06} mais stéréo et légère réverbération $\rightarrow$ \Large{$\frac{17}{20}$} \normalsize \vspace{0.2cm}
    \item \texttt{08} $\rightarrow$ \texttt{07} avec une empreinte stéréo différente $\rightarrow$ \Large{$\frac{17}{20}$} \normalsize \vspace{0.2cm}
    \item \texttt{09} $\rightarrow$ \texttt{01} en stéréo $\rightarrow$ \Large{$\frac{13}{20}$} \normalsize \vspace{0.2cm}
    \item \texttt{10} $\rightarrow$ pareil que \texttt{01} $\rightarrow$ \Large{$\frac{11}{20}$} \normalsize \vspace{0.2cm}
\end{itemize}


\vspace{1cm}
\subsection*{ZD 748}
\begin{itemize}
    \item \texttt{00} $\rightarrow$ bon synthétiseur LEAD mais basse qualité $\rightarrow$ \Large{$\frac{11}{20}$} \normalsize \vspace{0.2cm}
    \item \texttt{01} $\rightarrow$ \texttt{00} avec LFO stéréo + vibrato $\rightarrow$ \Large{$\frac{11}{20}$} \normalsize \vspace{0.2cm}
    \item \texttt{02} $\rightarrow$ \texttt{01} avec pan + beaucoup d'attack \\ et de release \\ $\rightarrow$ \Large{$\frac{15}{20}$} \normalsize \vspace{0.2cm}
    \item \texttt{03} $\rightarrow$ LEAD très sec, avec pan, sympa à utiliser \\ pour mélodie sèche $\rightarrow$ \Large{$\frac{15}{20}$} \normalsize \vspace{0.2cm}
    \item \texttt{04} $\rightarrow$ PAD mono LFO avec pan, attack longue, \\très léger mais release à réduire \\très bonne basse $\rightarrow$ \Large{$\frac{16}{20}$} \normalsize \vspace{0.2cm}
    \item \texttt{05} $\rightarrow$ attack du filtre violente mais sympa. Peut \\surtout être utilisé en mélodie secondaire \\ fait penser à FOREVER YOUNG $\rightarrow$ \Large{$\frac{16}{20}$} \normalsize \vspace{0.2cm}
    \item \texttt{06} $\rightarrow$ \texttt{00} mais en stéréo $\rightarrow$ \Large{$\frac{13}{20}$} \normalsize \vspace{0.2cm}
    \item \texttt{07} $\rightarrow$ \texttt{06} mais en dual mono $\rightarrow$ \Large{$\frac{10}{20}$} \normalsize \vspace{0.2cm}
\end{itemize}


\vspace{1cm}
\subsection*{ZD 749 $\rightarrow$ démo sympa}
\begin{itemize}
    \item \texttt{00} $\rightarrow$ la basse déchire, filtres à expérimenter \\orchestre complétement nul $\rightarrow$ \Large{$\frac{13}{20}$} \normalsize \vspace{0.2cm}
    \item \texttt{01} $\rightarrow$ grosse caisse à la ETS, très propre \\toms à la Rick Aschley, nettement moins  $\rightarrow$ \Large{$\frac{14}{20}$} \normalsize \vspace{0.2cm}
    \item \texttt{02} $\rightarrow$ bassline de Sweetest Perfection demo en stéréo \\aigu intéressant en LEAD $\rightarrow$ \Large{$\frac{15}{20}$} \normalsize \vspace{0.2cm}
    \item \texttt{03} $\rightarrow$ saxophone stéréo normal $\rightarrow$ \Large{$\frac{12}{20}$} \normalsize \vspace{0.2cm}
    \item \texttt{04} $\rightarrow$ \texttt{02} à l'octave inférieur, encore plus \\impessionnant $\rightarrow$ \Large{$\frac{17}{20}$} \normalsize \vspace{0.2cm}
    \item \texttt{05} $\rightarrow$ guitare mono chorus qui sonne bien $\rightarrow$ \Large{$\frac{13}{20}$} \normalsize \vspace{0.2cm}
    \item \texttt{06} $\rightarrow$ split avec les basses \texttt{00} et \\\texttt{05} $\rightarrow$ \Large{$\frac{14}{20}$} \normalsize \vspace{0.2cm}
    \item \texttt{07} $\rightarrow$ \texttt{05} avec stéréo $\rightarrow$ \Large{$\frac{16}{20}$} \normalsize \vspace{0.2cm}
    \item \texttt{08} $\rightarrow$ \texttt{07} plus complexe, avec un filtre \\à expérimenter $\rightarrow$ \Large{$\frac{15}{20}$} \normalsize \vspace{0.2cm}
        \subsubsection{Premier filtre pas mal}
        \textsf{Fc: 90  Q: 60   VLv: 04   VFc: 08   VQ: 16}
        \subsubsection{Deuxième filtre pas mal}
        \textsf{Fc: 70  Q: 60   Env: +27 (à expérimenter) \\ F: A: 24   H: 01   D: 32   S: 01   R: 12 \\  VLv: 04   VFc: 08   VQ: 16}
    \item \texttt{09} $\rightarrow$ guitare avec LFO en pan, pas mal \\du tout pourtant, bon effet stéréo en \\arrière plan $\rightarrow$ \Large{$\frac{15}{20}$} \normalsize \vspace{0.2cm}
    \item \texttt{10} $\rightarrow$ \texttt{02} et \texttt{07}, propre et utilisable $\rightarrow$ \Large{$\frac{17}{20}$} \normalsize \vspace{0.2cm}
\end{itemize}


\vspace{1cm}
\subsection*{ZD 700 $\rightarrow$ piano (seuls les presets notables seront abordés)}
\begin{itemize}
    \item \texttt{08} $\rightarrow$ piano sec dont on peut exploiter le filtre $\rightarrow$ \Large{$\frac{13}{20}$} \normalsize \vspace{0.2cm}
    \item \texttt{11} $\rightarrow$ sonne comme un synthétiseur, beaucoup de filtres, \\LFO stéréo, Fc  $\rightarrow$ \Large{$\frac{10}{20}$} \normalsize \vspace{0.2cm}
    \item \texttt{13} $\rightarrow$ arpège pas mal, à adapter en stéréo $\rightarrow$ \Large{$\frac{14}{20}$} \normalsize \vspace{0.2cm}
    \item \texttt{14} $\rightarrow$ arpège sec, accord tierce, usage limité $\rightarrow$ \Large{$\frac{14}{20}$} \normalsize \vspace{0.2cm}
    \item \texttt{21} $\rightarrow$ piano double octave, à retravailler $\rightarrow$ \Large{$\frac{15}{20}$} \normalsize \vspace{0.2cm}
\end{itemize}


\vspace{1cm}
\subsection*{ZD 701 $\rightarrow$ strings}
\begin{itemize}
    \item \texttt{00} $\rightarrow$ \textsf{arco strings}, stéréo $\rightarrow$ \Large{$\frac{12}{20}$} \normalsize \vspace{0.2cm}
    \item \texttt{01} $\rightarrow$ \texttt{00} mais en mono avec pan $\rightarrow$ \Large{$\frac{10}{20}$} \normalsize \vspace{0.2cm}
    \item \texttt{02} $\rightarrow$ \textsf{arco strings}, mono $\rightarrow$ \Large{$\frac{09}{20}$} \normalsize \vspace{0.2cm}
    \item \texttt{03} $\rightarrow$ \texttt{02} avec LFO en pan léger $\rightarrow$ \Large{$\frac{10}{20}$} \normalsize \vspace{0.2cm}
    \item \texttt{04} $\rightarrow$ \texttt{02} avec pan selon vélocité $\rightarrow$ \Large{$\frac{10}{20}$} \normalsize \vspace{0.2cm}
    \item \texttt{05} $\rightarrow$ \texttt{02} avec attack en fonction de la vélocité $\rightarrow$ \Large{$\frac{11}{20}$} \normalsize \vspace{0.2cm}
    \item \texttt{06} $\rightarrow$ \texttt{02} avec filtre $\rightarrow$ \Large{$\frac{10}{20}$} \normalsize \vspace{0.2cm}
    \item \texttt{07} $\rightarrow$ \texttt{02} PAD avec gros filtre doux $\rightarrow$ \Large{$\frac{12}{20}$} \normalsize \vspace{0.2cm}
    \item \texttt{08} $\rightarrow$ \texttt{02} avec filtre et LFO pan lent $\rightarrow$ \Large{$\frac{11}{20}$} \normalsize \vspace{0.2cm}
    \item \texttt{09} $\rightarrow$ \texttt{08} avec LFO plus rapide $\rightarrow$ \Large{$\frac{10}{20}$} \normalsize \vspace{0.2cm}
    \item \texttt{10} $\rightarrow$ arpège bizarre à la Britney Spears $\rightarrow$ \Large{$\frac{08}{20}$} \normalsize \vspace{0.2cm}
    \item \texttt{11} $\rightarrow$ arpège film d'horreur $\rightarrow$ \Large{$\frac{08}{20}$} \normalsize \vspace{0.2cm}
    \item \texttt{12} $\rightarrow$ arpège filtré, ressemble à Photographic \\à retravailler mais potentiel $\rightarrow$ \Large{$\frac{13}{20}$} \normalsize \vspace{0.2cm}
    \item \texttt{13} $\rightarrow$ \texttt{02} avec aigu $\rightarrow$ \Large{$\frac{08}{20}$} \normalsize \vspace{0.2cm}
    \item \texttt{14} $\rightarrow$ \texttt{13} avec une note différente $\rightarrow$ \Large{$\frac{08}{20}$} \normalsize \vspace{0.2cm}
    \item \texttt{15} $\rightarrow$ identique à \texttt{14}, inutile $\rightarrow$ \Large{$\frac{08}{20}$} \normalsize \vspace{0.2cm}
    \item \texttt{16} $\rightarrow$ avec forte vélocité, bruit métallique \\ avec faible vélocité, sonne comme \texttt{02} $\rightarrow$ \Large{$\frac{12}{20}$} \normalsize \vspace{0.2cm}
    \item \texttt{17} $\rightarrow$ \textsl{02} sans grande différence $\rightarrow$ \Large{$\frac{08}{20}$} \normalsize \vspace{0.2cm}
    \item \texttt{18} $\rightarrow$ \texttt{17} avec un peu de LFO en pan $\rightarrow$ \Large{$\frac{09}{20}$} \normalsize \vspace{0.2cm}
    \item \texttt{19} $\rightarrow$ \texttt{02} avec filtre plus lo-fi donc métallique $\rightarrow$ \Large{$\frac{13}{20}$} \normalsize \vspace{0.2cm}
    \item \texttt{20} $\rightarrow$ LEAD sec et métallique (10 kHz) \\peut faire des basses bien robotique\\ou rajouter des détails $\rightarrow$ \Large{$\frac{15}{20}$} \normalsize \vspace{0.2cm}
    \item \texttt{21} $\rightarrow$ \texttt{20} en long et LFO en pan $\rightarrow$ \Large{$\frac{15}{20}$} \normalsize \vspace{0.2cm}
    \item \texttt{22} $\rightarrow$ \texttt{21} filtré bas $\rightarrow$ \Large{$\frac{14}{20}$} \normalsize \vspace{0.2cm}
    \item \texttt{23} $\rightarrow$ \texttt{20} mais moins sec, plus métallique,\\on entend bien le filtre\\multi usage $\rightarrow$ \Large{$\frac{16}{20}$} \normalsize \vspace{0.2cm}
    \item \texttt{24} $\rightarrow$ \texttt{21} sans LFO $\rightarrow$ \Large{$\frac{10}{20}$} \normalsize \vspace{0.2cm}
    \item \texttt{25} $\rightarrow$ \texttt{02} en stéréo $\rightarrow$ \Large{$\frac{10}{20}$} \normalsize \vspace{0.2cm}
\end{itemize}


\vspace{1cm}
\subsection*{ZD 702 $\rightarrow$ guitare électrique et batterie style KRAFTWERK} 
\begin{itemize}
    \item \texttt{00} $\rightarrow$ guitare mono rock du style des années 70 $\rightarrow$ \Large{$\frac{13}{20}$} \normalsize \vspace{0.2cm}
    \item \texttt{01} $\rightarrow$ \texttt{00} en stéréo $\rightarrow$ \Large{$\frac{16}{20}$} \normalsize \vspace{0.2cm}
    \item \texttt{02} $\rightarrow$ \texttt{01} avec plus de son et réverbération légère $\rightarrow$ \Large{$\frac{17}{20}$} \normalsize \vspace{0.2cm}
    \item \texttt{03} $\rightarrow$ comme \texttt{01} $\rightarrow$ \Large{$\frac{16}{20}$} \normalsize \vspace{0.2cm}
    \item \texttt{04} $\rightarrow$ \texttt{01} en plus percutant $\rightarrow$ \Large{$\frac{17}{20}$} \normalsize \vspace{0.2cm}
    \item \texttt{05} $\rightarrow$ \texttt{04} avec release plus longue $\rightarrow$ \Large{$\frac{17}{20}$} \normalsize \vspace{0.2cm}
    \item \texttt{06} $\rightarrow$ \texttt{01} harmonisé à l'octave $\rightarrow$ \Large{$\frac{13}{20}$} \normalsize \vspace{0.2cm}
    \item \texttt{07} $\rightarrow$ \texttt{06} avec une meilleure empreinte stéréo $\rightarrow$ \Large{$\frac{14}{20}$} \normalsize \vspace{0.2cm}
    \item \texttt{08} $\rightarrow$ guitare transformée en PAD métallique \\LEAD et BASS propre \\utile pour composer car basses faciles à faire $\rightarrow$ \Large{$\frac{18}{20}$} \normalsize \vspace{0.2cm}
    \item \texttt{09} $\rightarrow$ \texttt{08} avec vibrato $\rightarrow$ \Large{$\frac{18}{20}$} \normalsize \vspace{0.2cm}
    \item \texttt{10} $\rightarrow$ arpège lent de \texttt{09} $\rightarrow$ \Large{$\frac{13}{20}$} \normalsize \vspace{0.2cm}
    \item \texttt{11} $\rightarrow$ stratocaster et basse. BASS percussive mais \\pas slap, très bien $\rightarrow$ \Large{$\frac{15}{20}$} \normalsize \vspace{0.2cm}
    \item \texttt{12} $\rightarrow$ \texttt{11} mais avec plus de son comme \texttt{02} $\rightarrow$ \Large{$\frac{16}{20}$} \normalsize \vspace{0.2cm}
    \item \texttt{13} $\rightarrow$ \texttt{11} avec une stéréo différente $\rightarrow$ \Large{$\frac{15}{20}$} \normalsize \vspace{0.2cm}
    \item \texttt{14} $\rightarrow$ \texttt{11} avec arpège sur la basse. $\rightarrow$ \Large{$\frac{14}{20}$} \normalsize \vspace{0.2cm}
    \item \texttt{15} $\rightarrow$ batterie style KRAFTWERK-BC \\ $\hookrightarrow$ peut aussi servir d'effet percussif $\rightarrow$ \Large{$\frac{17}{20}$} \normalsize \vspace{0.2cm}
    \item \texttt{16} $\rightarrow$ \texttt{15} la vélocité change au pitch \\ $\rightarrow$ \Large{$\frac{16}{20}$} \normalsize \vspace{0.2cm}
    \item \texttt{17} $\rightarrow$ \texttt{15} mais avec des effets, change \\légerement le son mais très intéressant \\ $\hookrightarrow$ très Kervokian $\rightarrow$ \Large{$\frac{17}{20}$} \normalsize \vspace{0.2cm}
    \item \texttt{18} $\rightarrow$ \texttt{17} avec effet de \texttt{16} $\rightarrow$ \Large{$\frac{16}{20}$} \normalsize \vspace{0.2cm}
    \item \texttt{19} $\rightarrow$ \texttt{17} légerement différent. Batterie de Music Non Stop 1983 $\rightarrow$ \Large{$\frac{17}{20}$} \normalsize \vspace{0.2cm}
    \item \texttt{20} $\rightarrow$ \texttt{00} + \texttt{15} $\rightarrow$ \Large{$\frac{12}{20}$} \normalsize \vspace{0.2cm}
    \item \texttt{21} $\rightarrow$ \texttt{17} + \texttt{02} $\rightarrow$ \Large{$\frac{14}{20}$} \normalsize \vspace{0.2cm}
    \item \texttt{22} $\rightarrow$ \texttt{15} + basse $\rightarrow$ \Large{$\frac{12}{20}$} \normalsize \vspace{0.2cm}
    \item \texttt{23} $\rightarrow$ \texttt{22} avec pitch affecté par la vélocité $\rightarrow$ \Large{$\frac{13}{20}$} \normalsize \vspace{0.2cm}
    \item \texttt{24} $\rightarrow$ \texttt{17} + basse $\rightarrow$ \Large{$\frac{14}{20}$} \normalsize \vspace{0.2cm}
    \item \texttt{25} $\rightarrow$ \texttt{19} + basse $\rightarrow$ \Large{$\frac{14}{20}$} \normalsize \vspace{0.2cm}
\end{itemize}


\vspace{1cm}
\subsection*{ZD 703} 
\begin{itemize}
    \item \texttt{00} $\rightarrow$ orgue très utile en accompagnement  \\ $\hookrightarrow$ style Lie To Me stéréo \\ $\hookrightarrow$ synthétisé $\rightarrow$ \Large{$\frac{17}{20}$} \normalsize \vspace{0.2cm}
    \item \texttt{01, 02} $\rightarrow$ pareil que \texttt{00} $\rightarrow$ \Large{$\frac{17}{20}$} \normalsize \vspace{0.2cm}
    \item \texttt{04} $\rightarrow$ \texttt{00} avec chorus $\rightarrow$ \Large{$\frac{16}{20}$} \normalsize \vspace{0.2cm}
    \item \texttt{05} $\rightarrow$ \texttt{00} harmonisé à l'octave $\rightarrow$ \Large{$\frac{13}{20}$} \normalsize \vspace{0.2cm}
    \item \texttt{06} $\rightarrow$ basse sèche et acide sensée imiter \\ un pizzicato $\rightarrow$ \Large{$\frac{11}{20}$} \normalsize \vspace{0.2cm}
    \item \texttt{07} $\rightarrow$ \texttt{06} + \texttt{04} $\rightarrow$ \Large{$\frac{10}{20}$} \normalsize \vspace{0.2cm}
    \item \texttt{08} $\rightarrow$ même remarque que \texttt{07} $\rightarrow$ \Large{$\frac{10}{20}$} \normalsize \vspace{0.2cm}
    \item \texttt{09} $\rightarrow$ basse inutile $\rightarrow$ \Large{$\frac{09}{20}$} \normalsize \vspace{0.2cm}
    \item \texttt{10} $\rightarrow$ \texttt{08} mais arpège sur la basse $\rightarrow$ \Large{$\frac{11}{20}$} \normalsize \vspace{0.2cm}
    \item \texttt{11} $\rightarrow$ \texttt{00} avec filtre en attack $\rightarrow$ \Large{$\frac{16}{20}$} \normalsize \vspace{0.2cm}
    \item \texttt{12} $\rightarrow$ \texttt{11} mais avec plus d'harmoniques \\ aigues, moins PAD, très PPG $\rightarrow$ \Large{$\frac{10}{20}$} \normalsize \vspace{0.2cm}
    \item \texttt{13} $\rightarrow$ \texttt{12} mais le filtre dépend de la vélocité $\rightarrow$ \Large{$\frac{10}{20}$} \normalsize \vspace{0.2cm}
\end{itemize}


\vspace{1cm}
\subsection*{ZD 704 - Cuivres à retoucher}
\begin{itemize}
    \item Globalement, des cuivres bien timbrés, stéréo pour la plupart. \\ Pas ouf séparément, mais peuvent bien sonner avec la bonne \\ musique et les bons filtres.
\end{itemize}


\vspace{1cm}
\subsection*{ZD 705 - French Horns}
\begin{itemize}
    \item Les French Horns sonnent bien, mais moins que ceux de EWHO. \\ En mezzo forte. Stéréo, filtres.
\end{itemize}


\vspace{1cm}
\subsection*{ZD 707 - Ch\oe urs}
\begin{itemize}
    \item \texttt{02} $\rightarrow$ ch\oe urs mixtes. Ressemble à ETS.\\ Basse incroyable $\rightarrow$ \Large{$\frac{17}{20}$} \normalsize \vspace{0.2cm}
    \item \texttt{09} $\rightarrow$ même observations pour des sons différents $\rightarrow$ \Large{$\frac{17}{20}$} \normalsize \vspace{0.2cm}10
    \item \texttt{11} $\rightarrow$ strings synthétisées à bien utiliser $\rightarrow$ \Large{$\frac{15}{20}$} \normalsize \vspace{0.2cm}
    \item \texttt{16} $\rightarrow$ \texttt{16} mais encore plus synthétisé. \\ Peut servir en PAD ou en LEAD $\rightarrow$ \Large{$\frac{16}{20}$} \normalsize \vspace{0.2cm}
\end{itemize}


\vspace{1cm}
\subsection*{ZD 708 - SP 12}
\begin{itemize}
    \item \texttt{00} $\rightarrow$ basse pas géniale $\rightarrow$ \Large{$\frac{09}{20}$} \normalsize \vspace{0.2cm}
    \item \texttt{01} $\rightarrow$ \texttt{00} mais un peu mieux $\rightarrow$ \Large{$\frac{10}{20}$} \normalsize \vspace{0.2cm}
    \item \texttt{02} $\rightarrow$ \texttt{01} avec pan $\rightarrow$ \Large{$\frac{10}{20}$} \normalsize \vspace{0.2cm}
    \item \texttt{03} $\rightarrow$ \texttt{02} avec plus de pan, peut \\ servir en arpège $\rightarrow$ \Large{$\frac{11}{20}$} \normalsize \vspace{0.2cm}
    \item \texttt{04} $\rightarrow$ \texttt{02} transformé en PAD $\rightarrow$ \Large{$\frac{12}{20}$} \normalsize \vspace{0.2cm}
    \item \texttt{05} $\rightarrow$ \texttt{04} encore plus PAD $\rightarrow$ \Large{$\frac{13}{20}$} \normalsize \vspace{0.2cm}
    \item \texttt{06} $\rightarrow$ \texttt{03} sec. Peut servir de basse mais \\ pas uniquement $\rightarrow$ \Large{$\frac{13}{20}$} \normalsize \vspace{0.2cm}
    \item \texttt{07} $\rightarrow$ basse avec batterie. Basse ressemblant à un \\ Jupiter 8 avec de la puissance brute. \\ Batterie typique d'un SP12$\rightarrow$ \Large{$\frac{16}{20}$} \normalsize \vspace{0.2cm}
    \item \texttt{08} $\rightarrow$ \texttt{07} avec meilleure stéréo et vélocité $\rightarrow$ \Large{$\frac{17}{20}$} \normalsize \vspace{0.2cm}
    \item \texttt{09} $\rightarrow$ \texttt{07} avec avec batterie moins aggressive. Basse molle \\ dans le style Minomoog, utilisable $\rightarrow$ \Large{$\frac{15}{20}$} \normalsize \vspace{0.2cm}
    \item \texttt{10} $\rightarrow$ \texttt{09} avec basse moins déphasée, plus propre dans le \\ style Big in Japan $\rightarrow$ \Large{$\frac{15}{20}$} \normalsize \vspace{0.2cm}
    \item \texttt{11} $\rightarrow$ batterie violente et réaliste, la meilleure jusqu'à \\ présent. Basse mélangée avec un LEAD. $\rightarrow$ \Large{$\frac{16}{20}$} \normalsize \vspace{0.2cm}
    \item \texttt{12} $\rightarrow$ batterie moyenne pan $\rightarrow$ \Large{$\frac{12}{20}$} \normalsize \vspace{0.2cm}
    \item \texttt{13} $\rightarrow$ tom à la Rick Aschley $\rightarrow$ \Large{$\frac{10}{20}$} \normalsize \vspace{0.2cm}
    \item \texttt{14} $\rightarrow$ \texttt{13} plus violent et de meilleure qualité $\rightarrow$ \Large{$\frac{12}{20}$} \normalsize \vspace{0.2cm}
    \item \texttt{15} $\rightarrow$ collection de caisses claires stéréo intéressante $\rightarrow$ \Large{$\frac{15}{20}$} \normalsize \vspace{0.2cm}
    \item \texttt{16} $\rightarrow$ template SP12 par défaut $\rightarrow$ \Large{$\frac{14}{20}$} \normalsize \vspace{0.2cm}
    \item \texttt{17} $\rightarrow$ basse et arpège utilisé sur la démo de \\ Liquid Stack. A utiliser avec prudence $\rightarrow$ \Large{$\frac{14}{20}$} \normalsize \vspace{0.2cm}
    \item \texttt{18} $\rightarrow$ basse type années 80 $\rightarrow$ \Large{$\frac{12}{20}$} \normalsize \vspace{0.2cm}
\end{itemize}


\vspace{1cm}
\subsection*{ZD 709}
\begin{itemize}
    \item \texttt{00} $\rightarrow$ une octave de guitare étouffée rock courte, \\ le reste est une guitare saturée. \\Bien pour les accords et les riffs $\rightarrow$ \Large{$\frac{15}{20}$} \normalsize \vspace{0.2cm}
    \item \texttt{01} $\rightarrow$ \texttt{00} mais ordre différent $\rightarrow$ \Large{$\frac{15}{20}$} \normalsize \vspace{0.2cm}
    \item \texttt{02} $\rightarrow$ batterie très rock, un peu new wave. \\ Multi-usage $\rightarrow$ \Large{$\frac{15}{20}$} \normalsize \vspace{0.2cm}
    \item \texttt{03} $\rightarrow$ \texttt{01} + \texttt{02} $\rightarrow$ \Large{$\frac{14}{20}$} \normalsize \vspace{0.2cm}
    \item \texttt{04} $\rightarrow$ \texttt{03} avec pan $\rightarrow$ \Large{$\frac{15}{20}$} \normalsize \vspace{0.2cm}
    \item \texttt{05} $\rightarrow$ \texttt{03} mais avec stéréo $\rightarrow$ \Large{$\frac{17}{20}$} \normalsize \vspace{0.2cm}
    \item \texttt{06} $\rightarrow$ comme \texttt{04} $\rightarrow$ \Large{$\frac{14}{20}$} \normalsize \vspace{0.2cm}
    \item \texttt{07} $\rightarrow$ guitare électrique saturée harmonisée à l'octave.\\ Bonne stéréo. Ressemble à Pleasure Little \\ Treasure. Bien pour effet et dynamisme.$\rightarrow$ \Large{$\frac{17}{20}$} \normalsize \vspace{0.2cm}
    \item \texttt{08} $\rightarrow$ mélange des instruments qui vont bien. \\ Très bonne stéréo. $\rightarrow$ \Large{$\frac{16}{20}$} \normalsize \vspace{0.2cm}
    \item \texttt{09} $\rightarrow$ moitié guitare étouffée, moitié guitare \\ saturée avec chorus $\rightarrow$ \Large{$\frac{16}{20}$} \normalsize \vspace{0.2cm}
    \item \texttt{10} $\rightarrow$ \texttt{09} sans chorus mais avec un gros filtre $\rightarrow$ \Large{$\frac{17}{20}$} \normalsize \vspace{0.2cm}
    \item \texttt{11} $\rightarrow$ guitare électrique saturée en accord $\rightarrow$ \Large{$\frac{12}{20}$} \normalsize \vspace{0.2cm}
    \item \texttt{12} $\rightarrow$ batterie pan $\rightarrow$ \Large{$\frac{12}{20}$} \normalsize \vspace{0.2cm}
    \item \texttt{13} $\rightarrow$ \texttt{12} mais sans pan. Mixage \\ différent $\rightarrow$ \Large{$\frac{12}{20}$} \normalsize \vspace{0.2cm}
    \item \texttt{14, 15} $\rightarrow$ déjà présent dans \texttt{ZD708}
\end{itemize}

\vspace{1cm}
\section*{English}

\end{document}

%   \texttt{00} $\rightarrow$  $\rightarrow$ \Large{$\frac{10}{20}$} \normalsize \vspace{0.2cm}